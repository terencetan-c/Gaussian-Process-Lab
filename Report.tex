\documentclass[11pt]{article}
\usepackage[a4paper, margin=20mm]{geometry}
 

\usepackage{amsmath}
\usepackage{physics}

\usepackage{graphicx}
\graphicspath{ {./figs/} }
\usepackage{subfig}

%\renewcommand{\baselinestretch}{2}
\usepackage{setspace}
\doublespacing
\usepackage{titlesec}

\titlespacing\section{0pt}{0pt plus 0pt minus 0pt}{0pt plus 2pt minus 2pt}
\titlespacing\subsection{0pt}{0pt plus 4pt minus 2pt}{0pt plus 2pt minus 2pt}
\titlespacing\subsubsection{0pt}{0pt plus 4pt minus 2pt}{0pt plus 2pt minus 2pt}



\begin{document}
\section{Exponentiated Quadratic}
\label{sec:Task 1}
The method outlined in this task is essentially a trial-and-error approach to obtaining a lower bound to prove false properties. It is obvious that not all false properties are guaranteed to be proven false, especially for properties with mostly negative output domains. This means that even with very large values of \emph{k}, which would also be computationally expensive, there is always a chance that this method will miss a false property.
\subsection{Average amount of time taken to generate the outputs}
\label{sec:131}
 




\end{document}

